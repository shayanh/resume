%%%%%%%%%%%%%%%%%%%%%%%%%%%%%%%%%%%%%%%%%
% Medium Length Professional CV
% LaTeX Template
% Version 2.0 (8/5/13)
%
% This template has been downloaded from:
% http://www.LaTeXTemplates.com
%
% Original author:
% Trey Hunner (http://www.treyhunner.com/)
%
% Important note:
% This template requires the resume.cls file to be in the same directory as the
% .tex file. The resume.cls file provides the resume style used for structuring the
% document.
%
%%%%%%%%%%%%%%%%%%%%%%%%%%%%%%%%%%%%%%%%%

%----------------------------------------------------------------------------------------
%	PACKAGES AND OTHER DOCUMENT CONFIGURATIONS
%----------------------------------------------------------------------------------------

\documentclass{resume} % Use the custom resume.cls style

\usepackage{multicol}
\usepackage{enumitem}
\usepackage{hyperref}
\hypersetup{}

\usepackage[left=0.70in,top=0.6in,right=0.70in,bottom=0.6in]{geometry} % Document margins

\name{Shayan Hosseini} % Your name
\address{No. 14, Tanhajou Street, Tehranpars 3rd Square, Tehran, Iran, 1653984753} % Your address
\address{+98-919-6421171 \\ \href{mailto:shayan.hosseiny@gmail.com}{\texttt{shayan.hosseiny@gmail.com}} \\ 
	\href{http://www.shayanh.ir}{\texttt{http://www.shayanh.ir}}} % Your phone number and email

\begin{document}
	
%----------------------------------------------------------------------------------------
%	INTERESTS SECTION
%----------------------------------------------------------------------------------------

\begin{rSection}{Research Interests}
	\begin{multicols}{2}
		\begin{itemize}[leftmargin=0mm]
			\item Distributed Systems
			\item Data Structures and Algorithms
			\item Computer Systems and Networks
			\item Programming Languages and Formal Methods
		\end{itemize}
	\end{multicols}
\end{rSection}

%----------------------------------------------------------------------------------------
%	EDUCATION SECTION
%----------------------------------------------------------------------------------------

\begin{rSection}{Education}
	
	{\bf \href{http://ut.ac.ir/en}{University of Tehran}}, Tehran, Iran \hfill 2015 - 2020 
	\\B.Sc. in Computer Engineering
	
	\begin{itemize}
		\item GPA: 17.38/20 
		%\item Related Courses: Design and Analysis of Algorithms (20/20), Data Structures and Algorithms (19.8/20),
		%Theory of Formal Languages and Automata (19.5/20), Algorithmic Graph Theory (18.5/20), Network Security (20/20), Computer Networks (18.1/20), Compilers and Programming Languages (19.55/20)
        \item Thesis: \textit{Verfication of Microservices}, Advisor: Prof. H. Hojjat
	\end{itemize}
	
	{\bf \href{http://www.helli.ir/}{Allameh Helli High School}}, Tehran, Iran \hfill 2011 - 2015
	\\\textit{\scriptsize Affiliated with the National Organization for the Development of Exceptional Talents (NODET)}
	\\Diploma in Mathematics and Physics Discipline.
	
\end{rSection}

%----------------------------------------------------------------------------------------
%	RESEARCH EXPERIENCE SECTION
%----------------------------------------------------------------------------------------

\begin{rSection}{Research Experience}
{\bf Research Intern at \href{https://www.mpi-inf.mpg.de/}{Max Planck Institute for Informatics}}, {\small Saarbr{\"u}cken, Germanry \hfill Summer 2019}
\\\textit{Under supervision of Prof. C. Lenzen}
	\\Member of algorithms and complexity group. I was working on the fault-tolerant routing problem with strong fault types.

{\bf Undergraduate Research Assistant at \href{http://ut.ac.ir/en}{University of Tehran}}, Tehran, Iran \hfill{2019}
\\\textit{Under supervision of Prof. B. Bahrak}
	\\Member of data analysis lab, working on various graph problems.

\end{rSection}

%----------------------------------------------------------------------------------------
%	AWARDS
%----------------------------------------------------------------------------------------

\begin{rSection}{Awards and Honors}
	
	{\bf \href{http://icpc.baylor.edu/}{ACM ICPC}}, International Collegiate Programming Contest by ACM.
	\begin{itemize}
		\item {\bf 56\textsuperscript{th} Team} in
		\href{https://icpc.baylor.edu/community/results-2017}{the 41\textsuperscript{th} ACM ICPC World Finals},
		South Dakota, USA\hfill 2017
		
		\item {\bf 3\textsuperscript{rd}, 2\textsuperscript{nd}, 2\textsuperscript{nd}, 9\textsuperscript{th} Team} in Regional Contests of ACM ICPC West Asia Region,
		Tehran site, respectively in
		\href{http://icpc.sharif.edu/acmicpc18/scoreboard/}{2018},
		\href{http://icpc.sharif.edu/acmicpc17/scoreboard/}{2017},
		\href{http://icpc.sharif.edu/acmicpc16/scoreboard/}{2016} and
		\href{http://icpc.sharif.edu/acmicpc15/scoreboard/}{2015}.
	\end{itemize}
	
	{\bf Silver Medal} in the 24\textsuperscript{th} \href{http://inoi.ir/}{Iranian National Olympiad in Informatics}\hfill 2014
	\begin{itemize}
		\item[] Medals awarded to 40 people after a year of competition between over 10000 students from all the country.
	\end{itemize}
	
	Member of \href{https://www.bmn.ir/}{{\bf National Elites Foundation}}, Iran \hfill 2014 - persent
	
\end{rSection}

%----------------------------------------------------------------------------------------
%	TEACHING
%----------------------------------------------------------------------------------------

\begin{rSection}{Teaching Experience}
	{\bf Teaching Assistant}
	\\\href{http://ut.ac.ir/en}{University of Tehran}
	\begin{itemize}
		\item {\bf Head TA}, Design and Analysis of Algorithms, H. Mahini \hfill 2018 - 2019 
		\item Computer Networks, A. Khonsari \hfill 2019 
		\item Engineering Probability and Statistics, B. Bahrak \hfill Fall 2018
		\item Advanced Programming, R. Khosravi, A. Sadeghi \hfill Fall 2017
		\item Discrete Mathematics, S. Mohammadi \hfill Spring 2017
		\item Data Structures and Algorithms, H. Feili \hfill Fall 2016
	\end{itemize}
	
	{\bf INOI and ACM ICPC related}
	\\Teaching topics in computer science including Algorithms, Data Structures,
	and Graph Theory to undergraduate students preparing for ACM ICPC as well as high school
	students preparing for INOI.
	\begin{itemize}
		\item \href{http://ut.ac.ir/en}{University of Tehran} \hfill 2017
		\item \href{http://inoi.ir/}{INOI Summer Camp}, at \href{http://ysc.ac.ir/}{Young Scholars Club} \hfill 2015
		\item \href{http://www.helli.ir/}{Allameh Helli High School} \hfill 2015
	\end{itemize}
\end{rSection}

%----------------------------------------------------------------------------------------
%	WORK
%----------------------------------------------------------------------------------------
\begin{rSection}{Professional Experience}
	{\bf Software Engineering Manager at \href{https://cafebazaar.ir/}{Cafebazaar}}, Tehran, Iran \hfill 2020
	\\ Member of the Hezardastan Cloud Services group. Leading a team of 10 people aiming to integrate all provided services into a unified product. Also, I was a member of the company's hiring committee.
	
	{\bf Technical Lead and Product Manager at \href{https://cafebazaar.ir/}{Cafebazaar}}, Tehran, Iran \hfill 2017 - 2019
	\\ Member of the infrastructure team. Aiming to provide high available and scalable services to other technical teams. Leading a team of 8 people inside this group. I'm one of the co-founders of Cafebazaar Cloud, which is a subsidiary of Cafebazaar that provides public cloud services to the other companies.
	\begin{itemize}
		\item Designed and developed a distributed Object storage service using Ceph, named \href{https://kise.roo.cloud/}{Kise}. Now it serves a large amount of data and responds to more than hundreds of requests per second. \href{https://cafebazaar.ir/}{Cafebazaar}, \href{https://divar.ir/}{Divar}, and \href{http://balad.ir/}{Balad} are using Kise heavily in production.
		\item Designed and developed a new architecture for our Content Delivery Network to use cache servers and IP Anycast network.
		\item Designed and developed a new software architecture for the CDN nodes to be dynamically configurable and multi-tenant.
		\item Designed and developed a smart DNS server for CDN traffic load balancing before we migrated to Anycast.
		\item As a Product Manager, responsible for:
		\begin{itemize}
			\item Determining the team's vision, OKRs, stories, and tasks.
			\item Collaborating with other technical teams and companies.
		\end{itemize}
		\item As a Tech Lead, responsible for deciding on technical problems, developing and maintaining these services.
	\end{itemize}
	
	{\bf Software Engineer at \href{https://cafebazaar.ir/}{Cafebazaar}}, Tehran, Iran \hfill 2015 - 2017
	\\ Cafebazaar is the most popular Android application store in Iran, with more than 35 million users.
	\begin{itemize}
		\item Member of the app acquisition team, working on application download and payment system.
		\item System owner of CDN and parts of Cafebazaar back-end. Responsible for code reviews and the systems to be available, scalable and maintainable.
		% \item As a Scrum Master, responsible for facilitating Scrum meetings and process conflicts resolution.
		\item Responsible for interviewing new software engineer candidates and improvement of the technical interview process.
	\end{itemize}
	
	{\bf Software Developer at \href{http://www.sabzgroup.com/}{Sabz Biomedicals}}, Tehran, Iran \hfill Summer 2015
	\\Developed software of Gel Documentation System (an instrument for high-quality visualization and recording of DNA bands). Implemented image processing part of the application for detecting DNA bands in a picture.
\end{rSection}

%\vspace{2mm}

%----------------------------------------------------------------------------------------
%	DEVELOPMENT
%----------------------------------------------------------------------------------------

%\begin{rSection}{Professional Development}
%	The following online courses were taken to acquire skills relevant to my work and research:
%	\begin{itemize}
%		\item {\bf Cloud Computing Concepts}, by University of Illinois at Urbana-Champaign, coursera.org
%		\item {\bf Bitcoin and Cryptocurrency Technologies}, by Princeton University, coursera.org
%		\item {\bf Game Theory}, by Stanford University and UBC, coursera.org
%	\end{itemize}
%\end{rSection}

%----------------------------------------------------------------------------------------
%	PROJECTS
%----------------------------------------------------------------------------------------
\begin{rSection}{Notable Projects}
    {\bf \href{https://github.com/shayanh/grpc-go-contracts}{gRPC Go Contracts}} \hfill 2020
    \\A design by contract library for the gRPC Go framework; To verify the communication of microservices by writing contracts for their RPCs.

%	{\bf \href{https://github.com/shayanh/Smoola}{Compiler Course Project}} \hfill 2018
%	\\Implemented a compiler for a new object-oriented programming language, named Smoola. This is done by compiling Smoola code to Java bytecode.
%
%	{\bf \href{https://github.com/shayanh/floodlight-sara-protocol}{Network Course Project}} \hfill 2018
%	\\Designed an algorithm for detecting topology of an unknown network; Used \href{https://github.com/floodlight/floodlight}{Floodlight} SDN OpenFlow Controller.
%
%	{\bf \href{https://github.com/shayanh/pintos}{Operating Systems Course Project}} \hfill 2017
%	\\Implemented the system call handler and an efficient CPU scheduler for communicating to many IO devices for \href{https://en.wikipedia.org/wiki/Pintos}{Pintos}.

	{\bf \href{https://github.com/shayanh/limoo}{Limoo}} \hfill 2017
	\\A desktop application that shows lyrics for the current playing song, server in \href{https://github.com/shayanh/limoo-server}{Go} and client in \href{https://github.com/shayanh/limoo}{Python}.

	{\bf \href{https://github.com/KhassTeam/Persian-CAPTCHA}{CAPTCHA Farsi}} \hfill 2011-2012
	\\Generating CAPTCHA in Persian by using image processing and calligraphy methods. Can be used in Persian websites. Awarded as Best Project in
	\href{https://www.helli.ir/portal/content/%D8%AA%D9%82%D8%AF%DB%8C%D8%B1-%D8%A7%D8%B2-%D8%AF%D8%A7%D9%86%D8%B4-%D8%A2%D9%85%D9%88%D8%B2%D8%A7%D9%86-%D8%A8%D8%B1%D8%AA%D8%B1%D9%BE%DA%98%D9%88%D9%87%D8%B4%DA%AF%D8%B1}
		{8th NODET Young Researchers Festival}.
\end{rSection}

%----------------------------------------------------------------------------------------
%	SKILLS
%----------------------------------------------------------------------------------------
\begin{rSection}{Skills and Qualities}

	{\bf Theoretical Background:}
	Expert in Design of Algorithms, Data Structures, and Discrete Mathematical fields such as Graph Theory and Combinatorics suggested by awards in ACM ICPC and Olympiad in Informatics.

	{\bf Languages and Tools:}
	Expert in C/C++, Go, Python, Bash-Scripting, Pascal, SQL, Git. Working knowledge in Java, Lua, Javascript, Verilog, \LaTeX.

	{\bf Software Engineering:}
	Familiar with different object-oriented design patterns, software development methodologies such as Agile and DevOps and, functional programming.

	{\bf Web Application Development:}
	Comfortable with Django, Flask, CSS3, HTML5 and ReactJS.

	{\bf Site Reliability Engineering:}
	Managed scalability and maintainability challenges in several web applications having thousands of requests per second.
	Expert Linux user and experienced in system administration.
	Experience in Ceph storage platform, Kubernetes, Docker, NGINX, Computer Networks.

	{\bf Languages:}
	English (Full professional proficiency), Persian (Native).

	{\bf Other:}
	Creative, self-motivated, eager to learn new things. Communicative and collaborative with the ability to work both independently and in a team.

\end{rSection}
	
\end{document}
