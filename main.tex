%%%%%%%%%%%%%%%%%%%%%%%%%%%%%%%%%%%%%%%%%
% Medium Length Professional CV
% LaTeX Template
% Version 2.0 (8/5/13)
%
% This template has been downloaded from:
% http://www.LaTeXTemplates.com
%
% Original author:
% Trey Hunner (http://www.treyhunner.com/)
%
% Important note:
% This template requires the resume.cls file to be in the same directory as the
% .tex file. The resume.cls file provides the resume style used for structuring the
% document.
%
%%%%%%%%%%%%%%%%%%%%%%%%%%%%%%%%%%%%%%%%%

%----------------------------------------------------------------------------------------
%	PACKAGES AND OTHER DOCUMENT CONFIGURATIONS
%----------------------------------------------------------------------------------------

\documentclass{resume} % Use the custom resume.cls style

\usepackage{multicol}
\usepackage{enumitem}
\usepackage{hyperref}
\hypersetup{}

\usepackage[left=0.50in,top=0.6in,right=0.50in,bottom=0.6in]{geometry} % Document margins

\name{Shayan Hosseini} % Your name
\address{X410D 2366 Main Mall, Vancouver, BC Canada V6T 1Z4} % Your address
\address{+1-604-841-1112 \\ \href{mailto:sshayanh@cs.ubc.ca}{\texttt{sshayanh@cs.ubc.ca}} \\ 
	\href{http://www.shayanh.ir}{\texttt{https://shayanh.com/}}} % Your phone number and email

\begin{document}
	
%----------------------------------------------------------------------------------------
%	EDUCATION SECTION
%----------------------------------------------------------------------------------------

\begin{rSection}{Education}
	{\bf \href{https://www.ubc.ca/}{University of British Columbia}}, Vancouver, BC, Canada \hfill 2021 - present
	\\M.Sc. in Computer Science
	
	{\bf \href{http://ut.ac.ir/en}{University of Tehran}}, Tehran, Iran \hfill 2015 - 2020 
	\\B.Sc. in Computer Engineering, GPA: 17.38/20 
    \\Thesis: \textit{Verfication of Microservices}
\end{rSection}

%----------------------------------------------------------------------------------------
%	RESEARCH EXPERIENCE SECTION
%----------------------------------------------------------------------------------------

\begin{rSection}{Research Experience}
	{\bf Research Assistant at the \href{https://www.ubc.ca/}{University of British Columbia}}, {Vancouver, BC, Canada \hfill 2021 - present}
	\\ \textit{Under supervision of Prof. I. Beschastnikh} \smallskip
	\vspace{-0.5em}
	\begin{itemize}[leftmargin=3mm]
		\setlength{\itemsep}{1pt}
		\setlength{\parskip}{0pt}
		\setlength{\parsep}{0pt}
		\renewcommand\labelitemi{$\cdot$}

		\item Working on the \href{https://github.com/UBC-NSS/pgo}{PGo} project, to compile verifiable models of the distributed system into implementations in Go.
		\item Proposed and implemented a solution to build verified fault-tolerant systems using PGo. Using it, I built a fault-tolerant distributed key-value store.
		\item Proposed a solution to build verified eventual consistent systems using PGo. Now working to build an eventually consistent key-value store using this solution.
	\end{itemize}

	{\bf Research Intern at \href{https://www.mpi-inf.mpg.de/}{Max Planck Institute for Informatics}}, {Saarbr{\"u}cken, Germanry \hfill Summer 2019}
	\\ \textit{Under supervision of Prof. C. Lenzen} \smallskip
	\vspace{-0.5em}
	\begin{itemize}[leftmargin=3mm]
		\setlength{\itemsep}{1pt}
		\setlength{\parskip}{0pt}
		\setlength{\parsep}{0pt}
		\renewcommand\labelitemi{$\cdot$}

		\item Working to make routing algorithms fault-tolerant by using an augmentation process. I proved the lower bound for the minimum number of edges 
		and vertices in the resulting augmented graph.
	\end{itemize}
\end{rSection}

%----------------------------------------------------------------------------------------
%	WORK
%----------------------------------------------------------------------------------------

\begin{rSection}{Professional Experience}
	{\bf Software Engineering Manager at \href{https://cafebazaar.ir/}{Cafebazaar}}, Tehran, Iran \hfill 2020
	\smallskip
	\vspace{-0.5em}
	\begin{itemize}[leftmargin=3mm]
		\setlength{\itemsep}{1pt}
		\setlength{\parskip}{0pt}
		\setlength{\parsep}{0pt}
		\renewcommand\labelitemi{$\cdot$}

		\item Leading a team of 10 people working to integrate all public cloud services into a unified product.
		\item Designed and implemented a billing system to automatically charge users for their usage of the services.
		\item Designed and implemented a monitoring and alerting system that allowed users to monitor their resources in the cloud.
	\end{itemize}
	
	{\bf Technical Lead and Product Manager at \href{https://cafebazaar.ir/}{Cafebazaar}}, Tehran, Iran \hfill 2017 - 2019
	\smallskip
	\vspace{-0.5em}
	\begin{itemize}[leftmargin=3mm]
		\setlength{\itemsep}{1pt}
		\setlength{\parskip}{0pt}
		\setlength{\parsep}{0pt}
		\renewcommand\labelitemi{$\cdot$}

		\item Co-founded and led a team of 8 engineers to provide highly available and scalable cloud service to other technical teams and companies (now rebranded as \href{https://sotoon.ir/}{Sotoon}).
		\item Started and developed a distributed object storage service using Ceph. It serves more than 50TB of data and responds to more than hundreds of requests per second in production.
		\item Scaled our existing content delivery network 50\% by developing a new architecture using cache servers.
		\item Designed and developed a new software architecture for the CDN nodes to be dynamically configurable and multi-tenant.% With this architecture, we were able to serve more than one customer in our network.
		% \item Decreased latency of our CDN 10\% by using a new load balancing architecture based on IP Anycast.
	\end{itemize}
	
	{\bf Software Engineer at \href{https://cafebazaar.ir/}{Cafebazaar}}, Tehran, Iran \hfill 2015 - 2017
	\smallskip
	\vspace{-0.5em}
	\begin{itemize}[leftmargin=3mm]
		\setlength{\itemsep}{1pt}
		\setlength{\parskip}{0pt}
		\setlength{\parsep}{0pt}
		\renewcommand\labelitemi{$\cdot$}

		\item Developing Cafebazaar back-end codebase, which had more than 100,000 lines of Python code and 30 million users.
		\item System owner of Cafebazaar CDN and parts of the back-end. Managing these systems on more than 15 servers.
		\item Responsible for code reviews and system maintenance.
	\end{itemize}
	
\end{rSection}

%----------------------------------------------------------------------------------------
%	AWARDS
%----------------------------------------------------------------------------------------

\begin{rSection}{Awards and Honors}
	
	{\bf \href{http://icpc.baylor.edu/}{ICPC}}, International Collegiate Programming Contest
	\\The International Collegiate Programming Contest is the most prestigious programming contest for college students.
	\smallskip
	\vspace{-0.5em}
	\begin{itemize}[leftmargin=3mm]
		\setlength{\itemsep}{1pt}
		\setlength{\parskip}{1pt}
		\setlength{\parsep}{0pt}
		\renewcommand\labelitemi{$\cdot$}

		\item {\bf 56\textsuperscript{th} team} in
		\href{https://icpc.baylor.edu/community/results-2017}{the 41\textsuperscript{th} ACM ICPC World Finals},
		South Dakota, USA\hfill 2017
		\\ Top 0.3\% among more than 45,000 students from all over the world.
		
		\item {\bf 3\textsuperscript{rd} team} in \href{http://www.acmicpc-pacnw.org/scoreboard/2020/index1.html}{ICPC Pacific Northwest Regional Contest}
		\hfill 2021
		\\ Among more than 100 teams participating from the area of Pacific NW (including Washington, Oregon, N. California, British Columbia).

		\item {\bf 3\textsuperscript{rd}, 2\textsuperscript{nd}, 2\textsuperscript{nd}, 9\textsuperscript{th} team} in Regional Contests of ACM ICPC West Asia Region,
		Tehran site, respectively in
		\href{http://icpc.sharif.edu/acmicpc18/scoreboard/}{2018},
		\href{http://icpc.sharif.edu/acmicpc17/scoreboard/}{2017},
		\href{http://icpc.sharif.edu/acmicpc16/scoreboard/}{2016} and
		\href{http://icpc.sharif.edu/acmicpc15/scoreboard/}{2015}.
		\\ Top 1\% among more than 300 teams that participate in this contest every year.
	\end{itemize}
	
	{\bf Silver Medal} in the 24\textsuperscript{th} \href{http://inoi.ir/}{Iranian National Olympiad in Informatics}\hfill 2014
	\smallskip
	\\Top 0.16\% in the algorithmic programming contest among 10,000 participants in the country.

\end{rSection}

%----------------------------------------------------------------------------------------
%	PROJECTS
%----------------------------------------------------------------------------------------
\begin{rSection}{Notable Projects}
	{\bf \href{https://peydaa.ir/}{Peydaa}} \hfill 2021 - present
	\\ Peydaa is a non-profit platform to make transparency in the Iranian job market. It is a website that people can anonymously share their salary and experience of working in companies with others. I founded Peydaa myself, working as both product manager and developer, and over time, four more people joined the team remotely. Until now, more than 600 users have shared their salaries and experiences on Peydaa.

	{\bf \href{https://github.com/UBC-NSS/pgo}{PGo}} \hfill 2021 - present
	\\PGo compiles verifiable formal specifications into Go implementations. I contributed to the PGo distributed runtime and built several distributed systems using PGo. 

	{\bf \href{https://github.com/DistributedClocks/tracing}{Tracing library}} \hfill 2021
	\\A light-weight library for manual distributed system tracing. Has been used for precise automatic grading in \href{https://www.cs.ubc.ca/~bestchai/teaching/cs416_2020w2/index.html}{CPSC 416 course}. I was the main designer and maintainer of this system and implemented most parts of it.

    {\bf \href{https://github.com/shayanh/grpc-go-contracts}{gRPC Go Contracts}} \hfill 2020
    \\A design by contract library for the gRPC Go framework; To verify the communication of microservices by writing contracts for their RPCs. I designed and implemented this project as a part of my BSc thesis.

	% {\bf \href{https://github.com/shayanh/limoo}{Limoo}} \hfill 2017
	% \\A desktop application that shows lyrics for the current playing song, server in \href{https://github.com/shayanh/limoo-server}{Go} and client in \href{https://github.com/shayanh/limoo}{Python}.

	% {\bf \href{https://github.com/KhassTeam/Persian-CAPTCHA}{CAPTCHA Farsi}} \hfill 2011-2012
	% \\Generating CAPTCHA in Persian by using image processing and calligraphy methods. Can be used in Persian websites. Awarded as Best Project in
	% \href{https://www.helli.ir/portal/content/%D8%AA%D9%82%D8%AF%DB%8C%D8%B1-%D8%A7%D8%B2-%D8%AF%D8%A7%D9%86%D8%B4-%D8%A2%D9%85%D9%88%D8%B2%D8%A7%D9%86-%D8%A8%D8%B1%D8%AA%D8%B1%D9%BE%DA%98%D9%88%D9%87%D8%B4%DA%AF%D8%B1}
	% 	{8th NODET Young Researchers Festival}.
\end{rSection}

%----------------------------------------------------------------------------------------
%	TEACHING
%----------------------------------------------------------------------------------------

\begin{rSection}{Teaching Experience}
	{\bf Teaching Assistant}
	\\\href{https://www.ubc.ca/}{University of British Columbia}
	\vspace{-0.5em}
	\begin{itemize}[leftmargin=3mm]
		\setlength{\itemsep}{1pt}
		\setlength{\parskip}{0pt}
		\setlength{\parsep}{0pt}
		\renewcommand\labelitemi{$\cdot$}

		\item CPSC 416: Distributed Systems, I. Beschastnikh \hfill Spring 2021
	\end{itemize}
	
	\href{http://ut.ac.ir/en}{University of Tehran}
	\vspace{-0.5em}
	\begin{itemize}[leftmargin=3mm]
		\setlength{\itemsep}{1pt}
		\setlength{\parskip}{0pt}
		\setlength{\parsep}{0pt}
		\renewcommand\labelitemi{$\cdot$}

		\item {\bf Head TA}, Design and Analysis of Algorithms, H. Mahini \hfill 2018 - 2019 
		\item Engineering Probability and Statistics, B. Bahrak \hfill Fall 2018, 2019
	\end{itemize}
	
	{\bf Informatics Olympiad and ICPC related}
	\\Teaching topics in computer science including Algorithms, Data Structures,
	and Graph Theory to undergraduate students preparing for ICPC as well as high school
	students preparing for Informatics Olympiad.

	\vspace{-0.5em}
	\begin{itemize}[leftmargin=3mm]
		\setlength{\itemsep}{1pt}
		\setlength{\parskip}{0pt}
		\setlength{\parsep}{0pt}
		\renewcommand\labelitemi{$\cdot$}

		\item \href{http://ut.ac.ir/en}{University of Tehran} \hfill 2017
		\item \href{http://inoi.ir/}{Iranian National Olympiad in Informatics Summer Camp} \hfill 2015
		\item \href{http://www.helli.ir/}{Allameh Helli High School} \hfill 2015
	\end{itemize}
\end{rSection}

%----------------------------------------------------------------------------------------
%	SKILLS
%----------------------------------------------------------------------------------------
\begin{rSection}{Skills and Qualities}

	{\bf Programming Languages and Tools:}
	Expert in Go, Python, Bash-Scripting, Git. Working knowledge in C/C++, SQL, Java, Lua, Javascript, Verilog, \LaTeX. \smallskip
	\\
	{\bf Software Engineering:}
	Familiar with different object-oriented design patterns, software development methodologies such as Agile and DevOps and, functional programming.\smallskip
	\\
	{\bf Web Application Development:}
	Comfortable with Django, Flask, CSS3, HTML5 and ReactJS.\smallskip
	\\
	{\bf Site Reliability Engineering:}
	Managed scalability and maintainability challenges in several web applications having thousands of requests per second.
	Expert Linux user and experienced in system administration.
	Experience with Ceph storage platform, Kubernetes, Docker, NGINX, Networks.\smallskip
	\\
	{\bf Theoretical Background:}
	Expert in design of algorithms, data structures, and discrete mathematical fields such as graph theory and combinatorics suggested by awards in ACM ICPC and Olympiad in Informatics.\smallskip
	\\
	{\bf Languages:}
	English (Full professional proficiency), Persian (Native).\smallskip

\end{rSection}
	
\end{document}
