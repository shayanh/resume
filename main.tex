%%%%%%%%%%%%%%%%%%%%%%%%%%%%%%%%%%%%%%%%%
% Medium Length Professional CV
% LaTeX Template
% Version 2.0 (8/5/13)
%
% This template has been downloaded from:
% http://www.LaTeXTemplates.com
%
% Original author:
% Trey Hunner (http://www.treyhunner.com/)
%
% Important note:
% This template requires the resume.cls file to be in the same directory as the
% .tex file. The resume.cls file provides the resume style used for structuring the
% document.
%
%%%%%%%%%%%%%%%%%%%%%%%%%%%%%%%%%%%%%%%%%

%----------------------------------------------------------------------------------------
%	PACKAGES AND OTHER DOCUMENT CONFIGURATIONS
%----------------------------------------------------------------------------------------

\documentclass{resume} % Use the custom resume.cls style

\usepackage{multicol}
\usepackage{enumitem}
\usepackage{hyperref}
\hypersetup{}

\usepackage[left=0.50in,top=0.6in,right=0.50in,bottom=0.6in]{geometry} % Document margins

\name{Shayan Hosseini} % Your name
\address{203-2150 W 39th Ave, Vancouver, BC V6M 1T5, Canada} % Your address
\address{+1-604-841-1112 \\ \href{mailto:sshayanh@cs.ubc.ca}{\texttt{sshayanh@cs.ubc.ca}} \\ 
	\href{http://www.shayanh.ir}{\texttt{https://shayanh.com/}}} % Your phone number and email

\begin{document}
	
%----------------------------------------------------------------------------------------
%	EDUCATION SECTION
%----------------------------------------------------------------------------------------

\begin{rSection}{Education}
	{\bf \href{https://www.ubc.ca/}{University of British Columbia}}, Vancouver, BC, Canada \hfill 2021 - present
	\\M.Sc. in Computer Science
	\\Advisor: Prof. Ivan Beschastnikh
	
	{\bf \href{http://ut.ac.ir/en}{University of Tehran}}, Tehran, Iran \hfill 2015 - 2020 
	\\B.Sc. in Computer Engineering, GPA: 17.38/20 
    \\Thesis: \textit{Verfication of Microservices}, Advisor: Prof. Hossein Hojjat
\end{rSection}

%----------------------------------------------------------------------------------------
%	RESEARCH EXPERIENCE SECTION
%----------------------------------------------------------------------------------------

\begin{rSection}{Research Experience}
	{\bf Research Assistant at the \href{https://www.ubc.ca/}{University of British Columbia}}, {Vancouver, BC, Canada \hfill 2021 - present}
	\\ \textit{Under supervision of Prof. I. Beschastnikh} \smallskip
	\\Member of the \href{https://systopia.cs.ubc.ca/}{computer systems group}. Working on \href{https://github.com/UBC-NSS/pgo}{PGo} project, to compile verifiable models of distributed system into implementions in Go.

	{\bf Research Intern at \href{https://www.mpi-inf.mpg.de/}{Max Planck Institute for Informatics}}, {Saarbr{\"u}cken, Germanry \hfill Summer 2019}
	\\ \textit{Under supervision of Prof. C. Lenzen} \smallskip
	\\Member of algorithms and complexity group. I was working on the fault-tolerant routing problem with strong fault types.

\end{rSection}

%----------------------------------------------------------------------------------------
%	WORK
%----------------------------------------------------------------------------------------

\begin{rSection}{Professional Experience}
	{\bf Founder of \href{https://peydaa.ir/}{Peydaa}} \hfill 2021 - present
	\smallskip
	\\ Peydaa is a non-profit platform to make transparency in the Iranian job market. It is a website that people can anonymously share their salary and experience of working in companies with others. I started Peydaa myself, working as product manager and developer and over time, four more people joined the team remotely. Until now, more than 600 users have shared their salaries and experiences on Peydaa.

	{\bf Software Engineering Manager at \href{https://cafebazaar.ir/}{Cafebazaar}}, Tehran, Iran \hfill 2020
	\smallskip
	\\ Member of \href{https://sotoon.ir/}{Sotoon}, providing cloud services to the other companies. I was leading a team of 10 people aiming to integrate all provided services into a unified product. Besides, I was a member of the company's hiring committee.
	
	{\bf Technical Lead and Product Manager at \href{https://cafebazaar.ir/}{Cafebazaar}}, Tehran, Iran \hfill 2017 - 2019
	\smallskip
	\\ Member of the infrastructure team. Aiming to provide high available and scalable services to other technical teams. Leading a team of 8 people inside this group. I was one of the co-founders of Cafebazaar Cloud, which is a subsidiary of Cafebazaar that provides public cloud services to the other companies (rebranded as \href{https://sotoon.ir/}{Sotoon} later).
	\vspace{-0.5em}
	\begin{itemize}[leftmargin=3mm]
		\setlength{\itemsep}{1pt}
		\setlength{\parskip}{0pt}
		\setlength{\parsep}{0pt}
		\renewcommand\labelitemi{$\cdot$}

		\item Designed and developed a distributed object storage service using Ceph, named \href{https://kise.roo.cloud/}{Kise}. It serves a large amount of data and responds to more than hundreds of requests per second in production.
		\item Designed and developed a new architecture for our Content Delivery Network to use cache servers and IP Anycast network.
		\item Designed and developed a new software architecture for the CDN nodes to be dynamically configurable and multi-tenant.
		% \item Designed and developed a smart DNS server for CDN traffic load balancing before we migrated to Anycast.
		\item As a Product Manager, responsible for determining the team's vision, OKRs, stories, and tasks.
		\item As a Tech Lead, responsible for deciding on technical problems, developing and maintaining these services.
	\end{itemize}
	
	{\bf Software Engineer at \href{https://cafebazaar.ir/}{Cafebazaar}}, Tehran, Iran \hfill 2015 - 2017
	\smallskip
	\\ Cafebazaar is the most popular Android application store in Iran, with more than 35 million users.
	\vspace{-0.5em}
	\begin{itemize}[leftmargin=3mm]
		\setlength{\itemsep}{1pt}
		\setlength{\parskip}{0pt}
		\setlength{\parsep}{0pt}
		\renewcommand\labelitemi{$\cdot$}

		\item Member of the app acquisition team, working on application download and payment system.
		\item System owner of CDN and parts of Cafebazaar back-end. Responsible for code reviews and the systems to be available, scalable and maintainable.
	\end{itemize}
	
\end{rSection}

%----------------------------------------------------------------------------------------
%	TEACHING
%----------------------------------------------------------------------------------------

\begin{rSection}{Teaching Experience}
	{\bf Teaching Assistant}
	\\\href{https://www.ubc.ca/}{University of British Columbia}
	\vspace{-0.5em}
	\begin{itemize}[leftmargin=3mm]
		\setlength{\itemsep}{1pt}
		\setlength{\parskip}{0pt}
		\setlength{\parsep}{0pt}
		\renewcommand\labelitemi{$\cdot$}

		\item CPSC 416: Distributed Systems, I. Beschastnikh \hfill Spring 2021
	\end{itemize}
	
	\href{http://ut.ac.ir/en}{University of Tehran}
	\vspace{-0.5em}
	\begin{itemize}[leftmargin=3mm]
		\setlength{\itemsep}{1pt}
		\setlength{\parskip}{0pt}
		\setlength{\parsep}{0pt}
		\renewcommand\labelitemi{$\cdot$}

		\item {\bf Head TA}, Design and Analysis of Algorithms, H. Mahini \hfill 2018 - 2019 
		\item Engineering Probability and Statistics, B. Bahrak \hfill Fall 2018, 2019
	\end{itemize}
	
	{\bf Informatics Olympiad and ICPC related}
	\\Teaching topics in computer science including Algorithms, Data Structures,
	and Graph Theory to undergraduate students preparing for ICPC as well as high school
	students preparing for Informatics Olympiad.

	\vspace{-0.5em}
	\begin{itemize}[leftmargin=3mm]
		\setlength{\itemsep}{1pt}
		\setlength{\parskip}{0pt}
		\setlength{\parsep}{0pt}
		\renewcommand\labelitemi{$\cdot$}

		\item \href{http://ut.ac.ir/en}{University of Tehran} \hfill 2017
		\item \href{http://inoi.ir/}{Iranian National Olympiad in Informatics Summer Camp} \hfill 2015
		\item \href{http://www.helli.ir/}{Allameh Helli High School} \hfill 2015
	\end{itemize}
\end{rSection}

%----------------------------------------------------------------------------------------
%	AWARDS
%----------------------------------------------------------------------------------------

\begin{rSection}{Awards and Honors}
	
	{\bf \href{http://icpc.baylor.edu/}{ICPC}}, International Collegiate Programming Contest
	\smallskip
	\\The International Collegiate Programming Contest is an algorithmic programming contest for college students.
	\vspace{-0.5em}
	\begin{itemize}[leftmargin=3mm]
		\setlength{\itemsep}{1pt}
		\setlength{\parskip}{0pt}
		\setlength{\parsep}{0pt}
		\renewcommand\labelitemi{$\cdot$}

		\item {\bf 56\textsuperscript{th} team} in
		\href{https://icpc.baylor.edu/community/results-2017}{the 41\textsuperscript{th} ACM ICPC World Finals},
		South Dakota, USA\hfill 2017
		
		\item {\bf 3\textsuperscript{rd} team} in \href{http://www.acmicpc-pacnw.org/scoreboard/2020/index1.html}{ICPC Pacific Northwest Regional Contest}
		\hfill 2021

		\item {\bf 3\textsuperscript{rd}, 2\textsuperscript{nd}, 2\textsuperscript{nd}, 9\textsuperscript{th} team} in Regional Contests of ACM ICPC West Asia Region,
		Tehran site, respectively in
		\href{http://icpc.sharif.edu/acmicpc18/scoreboard/}{2018},
		\href{http://icpc.sharif.edu/acmicpc17/scoreboard/}{2017},
		\href{http://icpc.sharif.edu/acmicpc16/scoreboard/}{2016} and
		\href{http://icpc.sharif.edu/acmicpc15/scoreboard/}{2015}.
	\end{itemize}
	
	{\bf Silver Medal} in the 24\textsuperscript{th} \href{http://inoi.ir/}{Iranian National Olympiad in Informatics}\hfill 2014
	\smallskip
	\\Iranian National Olympiad in Informatics is an annual computer-programming competition for high school students. Medals awarded to 40 people after a year of competition between over 10000 students from all the country.

\end{rSection}

%----------------------------------------------------------------------------------------
%	PROJECTS
%----------------------------------------------------------------------------------------
\begin{rSection}{Notable Open Source Projects}
	{\bf \href{https://github.com/UBC-NSS/pgo}{PGo}} \hfill 2021 - present
	\\PGo compiles verifiable formal speficiations into Go implementions. I worked on supporting fault-tolerant models and distributed state in PGo.

	{\bf \href{https://github.com/DistributedClocks/tracing}{Tracing library}} \hfill 2021
	\\A light-weight library for manual distributed system tracing. Has been used for precise automatic grading in \href{https://www.cs.ubc.ca/~bestchai/teaching/cs416_2020w2/index.html}{CPSC 416 course}.

    {\bf \href{https://github.com/shayanh/grpc-go-contracts}{gRPC Go Contracts}} \hfill 2020
    \\A design by contract library for the gRPC Go framework; To verify the communication of microservices by writing contracts for their RPCs.

	% {\bf \href{https://github.com/shayanh/limoo}{Limoo}} \hfill 2017
	% \\A desktop application that shows lyrics for the current playing song, server in \href{https://github.com/shayanh/limoo-server}{Go} and client in \href{https://github.com/shayanh/limoo}{Python}.

	{\bf \href{https://github.com/KhassTeam/Persian-CAPTCHA}{CAPTCHA Farsi}} \hfill 2011-2012
	\\Generating CAPTCHA in Persian by using image processing and calligraphy methods. Can be used in Persian websites. Awarded as Best Project in
	\href{https://www.helli.ir/portal/content/%D8%AA%D9%82%D8%AF%DB%8C%D8%B1-%D8%A7%D8%B2-%D8%AF%D8%A7%D9%86%D8%B4-%D8%A2%D9%85%D9%88%D8%B2%D8%A7%D9%86-%D8%A8%D8%B1%D8%AA%D8%B1%D9%BE%DA%98%D9%88%D9%87%D8%B4%DA%AF%D8%B1}
		{8th NODET Young Researchers Festival}.
\end{rSection}

%----------------------------------------------------------------------------------------
%	SKILLS
%----------------------------------------------------------------------------------------
\begin{rSection}{Skills and Qualities}

	{\bf Programming Languages and Tools:}
	Expert in Go, Python, Bash-Scripting, Git. Working knowledge in C/C++, SQL, Java, Lua, Javascript, Verilog, \LaTeX.

	{\bf Software Engineering:}
	Familiar with different object-oriented design patterns, software development methodologies such as Agile and DevOps and, functional programming.

	{\bf Web Application Development:}
	Comfortable with Django, Flask, CSS3, HTML5 and ReactJS.

	{\bf Site Reliability Engineering:}
	Managed scalability and maintainability challenges in several web applications having thousands of requests per second.
	Expert Linux user and experienced in system administration.
	Experience with Ceph storage platform, Kubernetes, Docker, NGINX, Networks.

	{\bf Theoretical Background:}
	Expert in design of algorithms, data structures, and discrete mathematical fields such as graph theory and combinatorics suggested by awards in ACM ICPC and Olympiad in Informatics.

	{\bf Languages:}
	English (Full professional proficiency), Persian (Native).

	{\bf Other:}
	Creative, self-motivated, eager to learn new things. Communicative and collaborative with the ability to work both independently and in a team.

\end{rSection}
	
\end{document}
