%%%%%%%%%%%%%%%%%%%%%%%%%%%%%%%%%%%%%%%%%
% Medium Length Professional CV
% LaTeX Template
% Version 2.0 (8/5/13)
%
% This template has been downloaded from:
% http://www.LaTeXTemplates.com
%
% Original author:
% Trey Hunner (http://www.treyhunner.com/)
%
% Important note:
% This template requires the resume.cls file to be in the same directory as the
% .tex file. The resume.cls file provides the resume style used for structuring the
% document.
%
%%%%%%%%%%%%%%%%%%%%%%%%%%%%%%%%%%%%%%%%%

%----------------------------------------------------------------------------------------
%	PACKAGES AND OTHER DOCUMENT CONFIGURATIONS
%----------------------------------------------------------------------------------------

\documentclass{resume} % Use the custom resume.cls style

\usepackage{multicol}
\usepackage{enumitem}
\usepackage{hyperref}
\hypersetup{}

% \usepackage[left=0.50in,top=0.6in,right=0.50in,bottom=0.6in]{geometry} % Document margins

\name{Shayan Hosseini} % Your name
\address{X410D 2366 Main Mall, Vancouver, BC Canada V6T 1Z4} % Your address
\address{+1-604-841-1112 \\
  \href{mailto:seshayanh@gmail.com}{\texttt{seshayanh@gmail.com}} \\ 
	\href{http://www.shayanh.ir}{\texttt{https://shayanh.com/}}} % Your phone number and email

\begin{document}
	
%----------------------------------------------------------------------------------------
%	EDUCATION SECTION
%----------------------------------------------------------------------------------------

\begin{rSection}{Education}
	{\bf \href{https://www.ubc.ca/}{University of British Columbia}}, Vancouver, BC, Canada
        \hfill 2021 - present
	\\M.Sc. in Computer Science
  \\\textit{Research topics}: I have been working broadly on distributed
  systems. My main research project has been building correct distributed
  systems using lightweight formal methods. I enjoy exploring different
  abstractions for making distributed systems better.

	{\bf \href{http://ut.ac.ir/en}{University of Tehran}}, Tehran, Iran \hfill 2015 - 2020 
	\\B.Sc. in Computer Engineering
  \\\textit{Thesis}: Verfication of Microservices
\end{rSection}

%----------------------------------------------------------------------------------------
%	PUBLICATIONS
%----------------------------------------------------------------------------------------

\begin{rSection}{Publications}
  {\bf \href{https://dl.acm.org/doi/10.1145/3575693.3575695}{Compiling Distributed System Models with PGo}} \hfill ASPLOS 2023
  \\ Finn Hackett, \textit{Shayan Hosseini}, Renato Costa, Matthew Do, and Ivan
  Beschastnikh
  \\ 28\textsuperscript{th} ACM International Conference on Architectural
  Support for Programming Languages and Operating Systems
\end{rSection}

%----------------------------------------------------------------------------------------
%	RESEARCH EXPERIENCE SECTION
%----------------------------------------------------------------------------------------

\begin{rSection}{Research Experience}
	{\bf Research Assistant at the \href{https://www.ubc.ca/}{University of British Columbia}},
    {Vancouver, BC, Canada \hfill 2021 - present}
    \\\textit{Advisor}: Ivan Beschastnikh \smallskip
	\vspace{-0.5em}
	\begin{itemize}[leftmargin=3mm]
		\setlength{\itemsep}{1pt}
		\setlength{\parskip}{0pt}
		\setlength{\parsep}{0pt}
		\renewcommand\labelitemi{$\cdot$}

    \item Building \href{https://github.com/DistCompiler/pgo}{PGo project}, a compiler for building correct distributed
      systems. PGo compiles verified models of distributed system models into
      implementations in Go.
    \item Created a distributed runtime library that provides the same model-level
      semantics on the implementation-level by employing distributed
      transactions and dependency injection.
    \item Made a verified distributed key-value store based on the Raft protocol
      by using PGo. Improving state-of-the-art $3\times$ on development effort and
      40\% on throughput.
	\item Built verified eventually consistent systems such as Conflict-free Replicated 
		Data Types by using PGo.
	\end{itemize}

  {\bf Research Intern at \href{https://www.microsoft.com/en-us/research/lab/microsoft-research-redmond/}{Microsoft Research Remond}},
    {Remote \hfill Summer 2022}
    \\\textit{Advisors}: Behnaz Arzani and Dan Crankshaw \smallskip
	\vspace{-0.5em}
	\begin{itemize}[leftmargin=3mm]
		\setlength{\itemsep}{1pt}
		\setlength{\parskip}{0pt}
		\setlength{\parsep}{0pt}
		\renewcommand\labelitemi{$\cdot$}

    \item Designed a system for automatic mitigation of faulty datacenter links
      by helping datacenter operators fixing the most critical links first.
    \item Improved failures of Azure's automatic mitigation of faulty datacenter links by 13\% based
      on the simulation results.
  \end{itemize}

	{\bf Research Intern at \href{https://www.mpi-inf.mpg.de/}{Max Planck Institute for Informatics}}, {Saarbr{\"u}cken, Germanry \hfill Summer 2019}
  \\\textit{Advisor}: Christoph Lenzen \smallskip
	\vspace{-0.5em}
	\begin{itemize}[leftmargin=3mm]
		\setlength{\itemsep}{1pt}
		\setlength{\parskip}{0pt}
		\setlength{\parsep}{0pt}
		\renewcommand\labelitemi{$\cdot$}

		\item Researching to make routing algorithms fault-tolerant by using an augmentation 
      process. Showed the lower bound for the minimum number of edges 
		  and vertices in the resulting augmented graph.
	\end{itemize}
\end{rSection}

%----------------------------------------------------------------------------------------
%	AWARDS
%----------------------------------------------------------------------------------------

\begin{rSection}{Awards and Honors}
	
	{\bf \href{http://icpc.baylor.edu/}{ICPC}}, International Collegiate Programming Contest
	\\The International Collegiate Programming Contest is the most prestigious programming 
  contest for college students.
	\smallskip
	\vspace{-0.5em}
	\begin{itemize}[leftmargin=3mm]
		\setlength{\itemsep}{1pt}
		\setlength{\parskip}{1pt}
		\setlength{\parsep}{0pt}
		\renewcommand\labelitemi{$\cdot$}

		\item {\bf World Finalist} in
		\href{https://icpc.global/community/results-2017}{the 41\textsuperscript{th} 
            ICPC World Finals},
		South Dakota, USA\hfill 2017
		\\ Top 0.3\% among more than 45,000 students from all over the world.
		
		\item {\bf 3\textsuperscript{rd} team} in
            \href{http://www.acmicpc-pacnw.org/scoreboard/2020/index1.html}
            {ICPC Pacific Northwest Regional Contest}
		\hfill 2021
		\\ Among more than 100 teams participating from the area of Pacific
    Northwest.

		\item {\bf 3\textsuperscript{rd}, 2\textsuperscript{nd}, 2\textsuperscript{nd} team} in 
      \href{https://icpc.ir/}{ICPC West Asia Region Regional Contest}, respectively in
      \href{http://icpc.sharif.edu/acmicpc18/scoreboard/}{2018},
      \href{http://icpc.sharif.edu/acmicpc17/scoreboard/}{2017},
      \href{http://icpc.sharif.edu/acmicpc16/scoreboard/}{2016}.
		\\ Top 1\% among more than 300 teams that participate in this contest every year.
	\end{itemize}
	
	{\bf Silver Medal} in the 24\textsuperscript{th} \href{http://inoi.ir/}
        {Iranian National Olympiad in Informatics}\hfill 2014
	\smallskip
	\\Top 0.16\% in the algorithmic programming contest among 10,000 participants 
    in the country.

\end{rSection}

%----------------------------------------------------------------------------------------
%	WORK
%----------------------------------------------------------------------------------------

\begin{rSection}{Professional Experience}
	{\bf Software Engineering Manager at \href{https://cafebazaar.ir/app?l=en}{Cafe Bazaar}}, Tehran, Iran \hfill 2020
	\smallskip
	\vspace{-0.5em}
	\begin{itemize}[leftmargin=3mm]
		\setlength{\itemsep}{1pt}
		\setlength{\parskip}{0pt}
		\setlength{\parsep}{0pt}
		\renewcommand\labelitemi{$\cdot$}

		\item Leading a team of 10 people working to integrate all public cloud services 
        	into a unified product.
		\item Built a billing system to automatically charge users for
            their usage of the services.
		\item Built a monitoring and alerting system that allowed users
            to monitor their resources in the cloud.
	\end{itemize}
	
	{\bf Technical Lead and Product Manager at \href{https://cafebazaar.ir/app?l=en}{Cafe Bazaar}}, Tehran, Iran \hfill 2017 - 2019
	\smallskip
	\vspace{-0.5em}
	\begin{itemize}[leftmargin=3mm]
		\setlength{\itemsep}{1pt}
		\setlength{\parskip}{0pt}
		\setlength{\parsep}{0pt}
		\renewcommand\labelitemi{$\cdot$}

		\item Co-founded and led a team of 8 engineers to provide highly available and 
      scalable cloud service to other technical teams and companies (later rebranded 
      as \href{https://sotoon.ir/}{Sotoon}).
		\item Started and developed a distributed object storage service using Ceph. 
      This service serves more than 50 TB of data and responds to more than hundreds of 
      requests per second in production.
		\item Scaled our existing content delivery network 50\% by developing a new 
      architecture using cache servers.
		\item Designed and developed a new software architecture for the CDN nodes to be 
      dynamically configurable and multi-tenant.% With this architecture, we were able to serve more than one customer in our network.
		% \item Decreased latency of our CDN 10\% by using a new load balancing architecture based on IP Anycast.
	\end{itemize}
	
	{\bf Software Engineer at \href{https://cafebazaar.ir/app?l=en}{Cafe Bazaar}}, Tehran, Iran \hfill 2015 - 2017
	\smallskip
	\vspace{-0.5em}
	\begin{itemize}[leftmargin=3mm]
		\setlength{\itemsep}{1pt}
		\setlength{\parskip}{0pt}
		\setlength{\parsep}{0pt}
		\renewcommand\labelitemi{$\cdot$}

		\item Developing Cafe Bazaar back-end codebase with more than 100,000 lines of 
      Python code and 30 million users.
		\item System owner of Cafe Bazaar CDN and parts of the back-end. Managing these systems 
      on more than 15 servers.
		\item Responsible for code reviews and system maintenance.
	\end{itemize}
	
\end{rSection}

%----------------------------------------------------------------------------------------
%	TEACHING
%----------------------------------------------------------------------------------------

\begin{rSection}{Teaching Experience}
	{\bf Teaching Assistant}
	\\\href{https://www.ubc.ca/}{University of British Columbia}
	\vspace{-0.5em}
	\begin{itemize}[leftmargin=3mm]
		\setlength{\itemsep}{1pt}
		\setlength{\parskip}{0pt}
		\setlength{\parsep}{0pt}
		\renewcommand\labelitemi{$\cdot$}

		\item CPSC 416: Distributed Systems, Ivan Beschastnikh \hfill Spring 2021
	\end{itemize}
	
	\href{http://ut.ac.ir/en}{University of Tehran}
	\vspace{-0.5em}
	\begin{itemize}[leftmargin=3mm]
		\setlength{\itemsep}{1pt}
		\setlength{\parskip}{0pt}
		\setlength{\parsep}{0pt}
		\renewcommand\labelitemi{$\cdot$}

		\item {\bf Head TA}, Design and Analysis of Algorithms, Hamid Mahini \hfill 2018 - 2019 
		\item Engineering Probability and Statistics, Behnam Bahrak \hfill Fall 2018, 2019
	\end{itemize}
	
	{\bf Informatics Olympiad and ICPC}
	\\Teaching topics in computer science including algorithms, data structures,
	and graph theory to undergraduate students preparing for ICPC as well as high school
	students preparing for Informatics Olympiad.

	\vspace{-0.5em}
	\begin{itemize}[leftmargin=3mm]
		\setlength{\itemsep}{1pt}
		\setlength{\parskip}{0pt}
		\setlength{\parsep}{0pt}
		\renewcommand\labelitemi{$\cdot$}

		\item \href{http://ut.ac.ir/en}{University of Tehran} \hfill 2017
		\item \href{http://inoi.ir/}{Iranian National Olympiad in Informatics Summer Camp} \hfill 2015
		\item \href{http://www.helli.ir/}{Allameh Helli High School} \hfill 2015
	\end{itemize}
\end{rSection}

%----------------------------------------------------------------------------------------
%	PROJECTS
%----------------------------------------------------------------------------------------
\begin{rSection}{Projects}
	{\bf \href{https://github.com/DistributedClocks/tracing}{Tracing Library}}
	\\ A light-weight library for manual distributed system tracing. Has been used for precise 
  automatic grading in the
  \href{https://www.cs.ubc.ca/~bestchai/teaching/cs416_2020w2/index.html}{CPSC 416 course}.  

  {\bf \href{https://github.com/shayanh/grpc-go-contracts}{gRPC Go Contracts}}
  \\ A design by contract library for the gRPC Go framework for runtime
  verification of microservices communication.

	{\bf \href{https://peydaa.ir/}{Peydaa}}
	\\ Peydaa is a non-profit platform to make transparency in the Iranian job market. 
  Peydaa is a website that people can anonymously share their salary and experience of working 
  in companies with others. More than 1000 users have shared their salaries and 
  experiences on Peydaa.
\end{rSection}

%----------------------------------------------------------------------------------------
%	TALKS
%----------------------------------------------------------------------------------------

\begin{rSection}{Talks}
  {\bf Compiling Distributed System Models Into Implementations with PGo}
	\vspace{-0.5em}
	\begin{itemize}[leftmargin=3mm]
		\setlength{\itemsep}{1pt}
		\setlength{\parskip}{0pt}
		\setlength{\parsep}{0pt}
		\renewcommand\labelitemi{$\cdot$}

		\item Microsoft Research, Networking Research Group \hfill November 2022
    \item TLA+ Conference \hfill September 2022
    \item AWS Automated Reasoning Group \hfill February 2022
    \item University of British Columbia \hfill February 2022
	\end{itemize}
\end{rSection}

\end{document}
